    \documentclass{article}
        
%% This file created automatically by CPBL's latexRegressionFile class
\usepackage{amsfonts} % Some substitions use mathbb
\usepackage[utf8]{inputenc}
\usepackage{lscape}
\usepackage{rotating}
\usepackage{relsize}
\usepackage{colortbl} %% handy for colored cells in tables, etc.
\usepackage{xcolor} %% For general, v powerful color handling: e.g simple coloured text.
%%\usepackage[svgnames]{xcolor} %% svgnames clashes with Beamer?
\usepackage{geometry}
\usepackage[colorlinks]{hyperref}

% If multirow is invoked, row labels will span the variable and its standard error.
\usepackage{multirow}
%\newcommand{\rowLabelWidth}{2cm}
%\newlength{\rowLabelWidth}
%\setlength{\rowLabelWidth}{2cm}

%% Make a series of capsules to format tables different ways:

\usepackage{cpblTables} % If you do not have this, just google for cpblTables.sty...
%\usepackage{cpblRef}
% Below I will uglyly stick varioius things from cpblRef or etc so that if that package is not included, they will at least be defined:
\ifdefined\cpblFigureTC
\else
% The "TC" in this command means there are two caption titles: one for TOC, one for figure.
\newcommand{\cpblFigureTC}[6]{%{\begin{figure}\centerline{\includegraphics{#1.eps}}
% Arguments: 1=filepath, 2=height or width declaration, 3=caption for TOC, 4=caption title, 5=caption details, 6=figlabel
\begin{figure}
  \begin{center}
    \includegraphics[#2]{#1}\caption[#3]{{\bf #4.} #5\label{fig:#6}\draftComment{\\{\bf label:} #6 {\bf file:} #1}}
  \end{center}
\end{figure}
}
\fi

\usepackage{xspace}
\ifdefined\draftComment
\else
\newcommand{\draftComment}[1]{{ \footnotesize\em\color{green} #1}\xspace}
\fi

%%  \useColourBoldForSignificance

\renewcommand{\ctDraftComment}[1]{{\sc\scriptsize ${\rm #1}$}} % Show draft-mode comments
\renewcommand{\ctDraftComment}[1]{{\sc\scriptsize #1}} % Show draft-mode comments

\newcommand{\texdocs}{/home/cpbl/bin/github/pystata/texdocs/}

        \geometry{verbose,letterpaper,tmargin=1cm,bmargin=1cm,lmargin=1cm,rmargin=1cm}
        %

\usepackage{chngcntr} %This is for counterwithin command, below. amsmath and numberwithin would be an alternative.
        \begin{document}
        \title{Regression results preview}\author{CPBL}\maketitle
%TOC:\tableofcontents
\listoftables
\listoffigures
\counterwithin{table}{section}
\counterwithin{figure}{section}
% Following would require amsmath, instead of chngcntr:
%\numberwithin{figure}{section}
%\numberwithin{table}{section}
        \pagestyle{empty}%%\newpage
        \begin{landscape}
    
\clearpage \newpage \clearpage

%\section{simple demo () ~[None]}
\newpage  \renewcommand{\sltrheadername}[1]{\multirow{2}{3cm}{\hspace{0pt}#1\vfill}}
\renewcommand{\sltrbheadername}[1]{\multirow{-2}{3cm}{\hspace{0pt}#1\vfill}}
\cpblTableCACSimple{lc|cc}{simple demo}{simple demo}{}{tab:simpledemo}{{\footnotesize\cpblColourLegend} }{\texdocs BPY-simpledemo-2014A}
\cpblTableCACSimpleTransposed{lp{3cm}*{10}{r}}{simple demo}{simple demo}{}{tab:simpledemo}{{\footnotesize\cpblColourLegend} }{\texdocs BPY-simpledemo-2014A}



\clearpage\newpage\end{landscape} End
\end{document}
